
\section{Newton's second law of motion, \acro{nsl}}
\label{sec:newton-second-law-motion}

Before writing \acro{nsl} in various forms, recall that~\cite{misner1973}
%
\begin{itemize}
\item acceleration is \emph{not} simply the temporal derivative of velocity, but its \emph{absolute} derivative: $\acc = \adt\vel = \oadt t\vel$, and
\item force and (linear) momentum are \emph{not} vectors, but rather \lingo{covectors} (one-forms).
\end{itemize}

Then, in \lingo{geometric notation}, \acro{nsl} becomes
%
\begin{equation*}
  \force = \adt\mom
  = \mass\fvec{\adt\vel}
  = \mass\fvec\acc
  = \mass\flat\oadt t\vel\,.
\end{equation*}
%
In \lingo{slot notation}:
%
\begin{equation*}
  \force = \mass\metric\vat{\eslot,\adt\vel}
  = \mass\metric\vat{\eslot,\acc}\,.
\end{equation*}
%
In \lingo{braket notation} (\aka\ Dirac's notation):
%
\begin{equation*}
  \bra\force = \mass\bra\acc\metric\,.
\end{equation*}
%
In \lingo{slot naming notation} (index notation) with \lingo{Einstein summation convention}, \acro{esc}:
%
\begin{equation*}
  \tcov\force i\tbcov i = \mass\tmetric ij\tvec\acc j\tbcov i\,.
\end{equation*}
%
In \lingo{Feynman's slash notation}, \acro{fsn}:
%
\begin{equation*}
  \tcov\force i\tbcov i = \mass\tmetric ij\tvec\acc j\tbcov i
  = \mass\tcov\acc i\tbcov i
  \implies
  \slashed\force = \mass\slashed\acc\,.
\end{equation*}
%
In \lingo{diagrammatic notation}:
%
\begin{equation*}
  {\rightarrow\bxtens\force}
  = {\rightarrow\bxtens\metric\leftarrow
  \leftarrow\bxtens\acc}\,.
\end{equation*}


\section{Gradient}
\label{sec:gradient}

To construct the gradient of a scalar field $\phi$, recall that the gradient of a tensor field $t$ is defined as
%
\begin{equation*}
  \grad t = \ograd t
  = \tbcov j\tprod\cder t{\tbvec j}
  = \tbcov j\tprod\ocder{\tbvec j}t\,.
\end{equation*}
%
Since a scalar field as $\phi$ can be regarded as a 0-rank tensor, apply the definition of gradient to $\phi$ to find
%
\begin{equation*}
  \grad\phi = \ograd\phi
  = \tbcov i\tprod\ocder{\tbvec i}\phi
  = \tbcov i\cpd i\phi
  = \slashed\partial\phi\,.
\end{equation*}
%
Note that $\grad\phi$ is a covector, by construction. So, if the gradient \emph{vector}, $\vgrad\phi$, is required, then $\grad\phi$ needs to be \scare{sharpened}:
%
\begin{equation*}
  \vgrad\phi = \scov{\grad\phi}
  = \scov{\slashed\partial\phi}\,.
\end{equation*}


\section{Divergence}
\label{sec:divergence}

Construct the divergence of a vector $v$.

Recall that, by definition, the divergence of a vector is the contraction of its gradient; \ie,
%
\begin{equation*}
  \vdiv v = \div v
  = \ocont\ograd v
  = \odiv v\,.
\end{equation*}
%
So, find first $\grad v$:
%
\begin{equation*}
  \grad v = \ograd v
  = \tbcov j\tprod\ocder{\tbvec j}\br{\tvec vi\tbvec i}
  = \br{\cpd j\tvec vi + \tchris ijk\tvec vk}\tbcov j\tprod\tbvec i\,.
\end{equation*}
%
Now, contract $\grad v$ by replacing the tensor product in $\grad v$ with the inner product: $\tbcov j\tprod\tbvec i = \tbcov j\iprod\tbvec i = \tkron ji$. This means that
%
\begin{align*}
  \cont{\grad v} &= \br{\cpd j\tvec vi + \tchris ijk\tvec vk}\tkron ji
  = \cpd j\tkron ji\tvec vi + \tkron ji\tchris ijk\tvec vk \,,\\
  &= \cpd i\tvec vi + \tchris iik\tvec vk \,.
\end{align*}
%
Finally, relabel the dummy $k$ to $j$ to arrive to our desired result
%
\begin{equation*}
  \vdiv v = \cpd i\tvec vi + \tchris iij\tvec vj \,.
\end{equation*}


\section{Advective mass transport}
\label{sec:advective-mass-transport}

Suppose a fluid of mass density $\dens$ flow with velocity $\vel$. Suppose advection be the only transport agent. Then, the advective flux is $\flux = \dens\vel$; while the evolution of the fluid mass,
%
\begin{equation*}
  0 = \cpd t\dens + \div{\dens\vel}
  = \cpd t\dens + \odiv{\br{\dens\vel}} \,.
\end{equation*}
%
Distribute the divergence over the $\dens\vel$ product using the product rule, replace the definition of the covariant derivative and then that of the material derivative to find
%
\begin{align*}
  0 &= \cpd t\dens + \vel\iprod\ograd\dens + \dens\ograd\iprod\vel
  = \cpd t\dens + \ocder\vel\dens + \dens\odiv\vel \,,\\
  &= \omdt t\dens + \dens\odiv\vel \,.
\end{align*}
%
Finally, replace the operators for decorations:
%
\begin{equation*}
  \mdt\dens + \dens\div\vel = 0\,.
\end{equation*}
%
Now, let us work out the coordinate form of the last equation. First, $\mdt\dens$:
%
\begin{equation*}
  \mdt\dens = \omdt t\dens
  = \cpd t\dens + \vel\iprod\ograd\dens \,.
\end{equation*}
%
Recall that $\ograd\dens = \tbcov i\tprod\ocder{\tbcov i}\dens = \tbcov i\cpd i\dens$. Thus,
%
\begin{equation*}
  \vel\iprod\ograd\dens = \tvec\vel j\tbvec j\iprod\tbcov i\cpd i\dens
  = \tvec\vel j\tkron ij\cpd i\dens
  = \tvec\vel i\cpd i\dens \,.
\end{equation*}
%
And hence
%
\begin{equation*}
  \mdt\dens = \cpd t\dens + \tvec\vel i\cpd i\dens \,.
\end{equation*}
%
On the other hand, using the results from the previous section, see that
%
\begin{equation*}
  \dens\odiv\vel = \dens\br{\cpd i\tvec\vel i + \tchris iij\tvec\vel j} \,.
\end{equation*}
%
Finally, putting the values for $\mdt\dens$ and $\dens\div\vel$ together, one has that $\mdt\dens + \dens\div\vel = 0$ becomes
%
\begin{equation*}
  \cpd t\dens + \tvec\vel i\cpd i\dens
  + \dens\br{\cpd i\tvec\vel i + \tchris iij\tvec\vel j}
  = 0\,.
\end{equation*}


\section{Physical components}
\label{sec:physical-components}

A physical component of a vector field has the same units as the field. Thus, a physical component of velocity has units $\si{m/s}$.

Consider a vector $v\in\reals^n$. Then, $v$ can be expanded in a \lingo{coordinate basis} $\tbvec i$ or in a \lingo{physical basis} $\pbvec i$ (unit vectors):
%
\begin{equation*}
  v = \tvec vi\tbvec i = \pvec vi\pbvec i\,.
\end{equation*}
%
Recall that unit (physical) basis vectors are constructed as
%
\begin{equation*}
  \pbvec i = \dfrac{\tbvec i}{\sqrt{\pmetric ii}}\,,
\end{equation*}
%
where the underlined indices $\underline{ii}$ prevent an unwanted trigger of \acro{esc}.

This means that
%
\begin{equation*}
  v = \tvec vi\tbvec i = \tvec vi\sqrt{\pmetric ii}\pbvec i\,,
\end{equation*}
%
which gives the relation between contravariant components and coordinate basis vectors and physical components and unit basis vectors.

On the other hand, consider a covector $v\u\in\reals^n$. Then, $u$ can be expanded in a \lingo{coordinate basis} $\tbcov i$ or in a \lingo{physical basis} $\pbcov i$ (unit vectors):
%
\begin{equation*}
  v = \tcov vi\tbcov i = \pcov vi\pbcov i\,.
\end{equation*}
%
Recall that unit (physical) basis covectors are constructed as
%
\begin{equation*}
  \pbcov i = \dfrac{\tbcov i}{\sqrt{\pimetric ii}}\,,
\end{equation*}
%
where the underlined indices $\underline{ii}$ prevent an unwanted trigger of \acro{esc}.

This means that
%
\begin{equation*}
  v = \tcov vi\tbcov i = \tcov vi\sqrt{\pimetric ii}\pbcov i\,,
\end{equation*}
%
which gives the relation between covariant components and coordinate basis vectors and physical components and unit basis vectors.


\section{Polar coordinates}
\label{sec:polar-coordinates}


\subsection{Metric in polar coordinates}
\label{subsec:metric-polar-coordinates}

Polar coordinates $\tuple{\cxpos,\cypos}$ are related to Cartesians $\tuple{\xpos,\ypos}$ \via\ the transformations:
%
\begin{equation*}
  \xpos = \cxpos\cos\cypos\,,
  \quad
  \ypos = \cxpos\sin\cypos\,.
\end{equation*}
%
\lingo{Jacob's matrix transformation} is given by
%
\begin{equation*}
  \jacob = \cpd i\tvec\pos{\hat\jmath}
  =
  \begin{pmatrix}
    \cpd\xpos\cxpos & \cpd\xpos\cypos \\
    \cpd\ypos\cxpos & \cpd\ypos\cypos \\
  \end{pmatrix}
  =
  \begin{pmatrix}
    \cos\cypos & -\cxpos\sin\cypos \\
    \sin\cypos & \cxpos\cos\cypos
  \end{pmatrix}
  \,;
\end{equation*}
%
while the \lingo{coefficients of the metric} (tensor) by
%
\begin{equation*}
  \metric = \jacob^T\jacob
  =
  \begin{pmatrix}
    1 & 0 \\
    0 & \cxpos^2 \\
  \end{pmatrix}
  \,.
\end{equation*}


\subsection{Kinetic energy during circular motion}
\label{subsec:kin-energy-circular-motion}

Consider a particle of mass $\mass$ orbiting a larger body. The particle position can be modeled by
%
\begin{equation*}
  \pos = \cxpos\tbvec\cxpos + \cypos\tbvec\cypos\,,
\end{equation*}
%
and hence its velocity by
%
\begin{equation*}
  \vel = \mdt\pos = \mdt\cxpos\tbvec\cxpos + \mdt\cypos\tbvec\cypos\,.
\end{equation*}
%
(Note that we are using local coordinates -- local components and local coordinate basis --, so we do the physics as we would do with Cartesians.)

Now, the particle kinetic energy is given by
%
\begin{equation*}
  2\energy = \mass\vel^2
  = \mass\vel\fvec\vel
  = \mass\metric\vat{\vel,\vel}\,,
\end{equation*}
%
which in local coordinates turns into
%
\begin{equation*}
  2\energy = \mass\tmetric ij\tvec\vel i\tvec\vel j\,,
\end{equation*}
%
with $i,j$ take the $\cxpos,\cypos$ values.

Expand then the indices and replace the values of the metric coefficients to find
%
\begin{align*}
  2\energy &= \mass\br{
    \tmetric 11\tvec\vel 1\tvec\vel 1
    + \tmetric 12\tvec\vel 1\tvec\vel 2
    + \tmetric 21\tvec\vel 2\tvec\vel 1
    + \tmetric 22\tvec\vel 2\tvec\vel 2
  } \,,\\
  &= \mass\br{\mdt\cxpos\mdt\cxpos + 0 + 0 + \cxpos^2\mdt\cypos\mdt\cypos}\,,
\end{align*}
%
which, after some algebra, becomes
%
\begin{equation*}
  2\energy = \mass\br{\mdt\cxpos\mdt\cxpos + \cxpos^2\mdt\cypos\mdt\cypos}\,.
\end{equation*}
%
The last equation is our final result.


\subsection{Velocity in circular motion}
\label{subsec:velocity-circular-motion}

Consider a particle orbiting a body. Let $\pos$ be the position of the particle at any time $t$. Then, its velocity is, by definition, $\vel = \mdt\pos$, which, in polar coordinates, in contravariant form, can be written as
%
\begin{equation*}
  \vel = \mdt\cxpos\tbvec\cxpos + \mdt\cypos\tbvec\cypos\,.
\end{equation*}
%
Notice that the basis is \emph{coordinate}, not \emph{unit}; \ie, the $\tbvec i$ have dimensions. To transform the components of $\vel$ into physical form, we use the metric:
%
\begin{equation*}
  \tvec\vel\cxpos
  = \mdt\cxpos\sqrt{\pmetric\cxpos\cxpos}\pbvec\cxpos
  = \mdt\cxpos\br{1}\pbvec\cxpos \,,
\end{equation*}
%
and
%
\begin{equation*}
  \tvec\vel\cypos
  = \mdt\cypos\sqrt{\pmetric\cypos\cypos}\pbvec\cypos
  = \mdt\cypos\sqrt{\cxpos^2}\pbvec\cypos
  = \mdt\cypos\cxpos\pbvec\cypos \,,
\end{equation*}
%
which leads to
%
\begin{equation*}
  \vel = \mdt\cxpos\pbvec\cxpos + \cxpos\mdt\cypos\pbvec\cypos \,.
\end{equation*}
%
This is the expression of velocity in \lingo{physical form} in polar coordinates.
