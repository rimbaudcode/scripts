
\section{Newton's second law of motion}
\label{sec:newton-second-law-motion}

Write Newton's second law of motion in various notations.~\cite{misner1973}

In \lingo{slot notation}:
%
\begin{equation*}
  \force = \adt\mom
  = \mass\fvec{\adt\vel}
  = \mass\fvec\acc
  = \mass\metric\vat{\eslot,\acc}\,.
\end{equation*}
%
In \lingo{braket notation} (\aka\ Dirac notation):
%
\begin{equation*}
  \bra\force = \mass\bra{\mass\acc}\metric\,.
\end{equation*}
%
In index notation with \lingo{Einstein summation convention}, \acro{esc}:
%
\begin{equation*}
  \tcov\force i\tbcov i = \mass\tmetric ij\tvec\acc j\tbcov i\,.
\end{equation*}
%
In \lingo{Feynman slash notation}, \acro{fsn}:
%
\begin{equation*}
  \tcov\force i\tbcov i = \mass\tmetric ij\tvec\acc j\tbcov i
  = \mass\tcov\acc i\tbcov i
  \implies
  \slashed\force = \mass\slashed\acc\,.
\end{equation*}
%
In diagrammatic notation:
%
\begin{equation*}
  {\rightarrow\bxtens\force}
  = {\rightarrow\bxtens\metric\leftarrow
  \leftarrow\bxtens\acc}\,.
\end{equation*}


\section{Navier-Stokes' equation}
\label{sec:nse}

Write Navier-Stokes' equation in various notations.~\cite{misner1973}


With decorations:
%
\begin{equation*}
  \adt\vel - \kkvisc\lap\vel = -\spvol\scov{\grad\press} + \grav\,.
\end{equation*}
%
\begin{equation*}
  \dt\vel + \cder\vel\vel - \kkvisc\lap\vel = -\spvol\scov{\grad\press} + \grav\,.
\end{equation*}
%
With operators:
%
\begin{equation*}
  \oadt t\vel - \kkvisc\olap\vel = -\spvol\oscov\ograd\press + \grav\,.
\end{equation*}
%
\begin{equation*}
  \cpd t\vel + \ocder\vel\vel - \kkvisc\olap\vel = -\spvol\oscov\ograd\press + \grav\,.
\end{equation*}
%
In braket notation:
%
\begin{equation*}
  \ket{\adt\vel} - \kkvisc\ket{\lap\vel} = -\spvol\imetric\ket{\grad\press} + \ket\grav\,,
\end{equation*}
%
where $\imetric$ represents the \lingo{inverse metric}.


\section{Gradient}
\label{sec:gradient}

Construct the gradient of a scalar field $\phi$.

Recall that a scalar field is a tensor field of rank 0 and, by definition, the gradient of a tensor field $t$ is
%
\begin{equation*}
  \grad t = \ograd t
  = \tbcov j\tprod\cder t{\tbvec j}
  = \tbcov j\tprod\ocder{\tbvec j}t\,.
\end{equation*}
%
Apply this definition to $\grad\phi$ to find
%
\begin{equation*}
  \grad\phi = \tbcov i\tprod\ocder{\tbvec i}\phi
  = \tbcov i\cpd i\phi
  = \slashed\partial\phi\,.
\end{equation*}
%
Note that $\grad\phi$ is a covector, by construction. So, if the gradient \emph{vector} is required, $\vgrad\phi$, then the covector $\grad\phi$ needs to be sharpen:
%
\begin{equation*}
  \vgrad\phi = \scov{\grad\phi}
  = \scov{\slashed\partial\phi}\,.
\end{equation*}


\section{Divergence}
\label{sec:divergence}

Construct the divergence of a vector $v$.

Recall that, by definition, the divergence of a vector is the contraction of its gradient; \ie,
%
\begin{equation*}
  \vdiv v = \cont\grad v
  = \cont\ograd v
  = \odiv v\,.
\end{equation*}
%
So, find first $\grad v$:
%
\begin{equation*}
  \grad v = \ograd v
  = \tbcov j\tprod\ocder{\tbvec j}\br{\tvec vi\tbvec i}
  = \br{\cpd j\tvec vi + \tchris ijk\tvec vk}\tbcov j\tprod\tbvec i\,.
\end{equation*}
%
Now, contract $\grad v$ by replacing the tensor product in $\grad v$ with the inner product: $\tbcov j\tprod\tbvec i = \tbcov j\iprod\tbvec i = \tkron ji$, which means that
%
\begin{align*}
  \cont\grad v &= \br{\cpd j\tvec vi + \tchris ijk\tvec vk}\tkron ji\,,\\
  &= \cpd j\tvec vi\tkron ji + \tchris ijk\tkron ji\tvec vk \,,\\
  &= \cpd j\tvec vj + \tchris jjk\tvec vk \,.
\end{align*}
%
Finally, rename the (dummy) $j$ indices to $i$ and then the $k$ indices to $j$:
%
\begin{equation*}
  \vdiv v = \cpd i\tvec vi + \tchris iij\tvec vj\,.
\end{equation*}
%
which gives the desired result.
