
\section{The geometric principle of physics laws}
\label{sec:geometric-principle-physics-laws}

Herein, the \lingo{geometric principle of physics laws} will permeate our way of doing physics:~\cite[chap. 1, p. iii]{thorne2013}
%
\begin{axiom}[Geometric principle]\label{axm:geometric-principle-physics-laws}
  The laws of physics must be expressible as geometric relationships between geometric objects, which represent physical quantities.
\end{axiom}
%
We will profit from this principle not only to express laws, but also to assess, derive, and interpret them.


\section{Balance law of a conserved quantity}
\label{sec:balance-law-conserved-quantity}

\begin{definition}
  Refer to a \lingo{continuum} as a region in space.
\end{definition}

\begin{observation}\label{obs:balance-law-conserved-quantity}
  Consider a conserved quantity move through a body. Suppose the body be represented as a \lingo{continuum} $\Omega$ and the quantity $\Phi$, of density $\phi$, be defined for all points in  $\Omega$. Choose now a control region $\omega$ in $\Omega$ bounded by a control surface $\bregion\omega$. Finally, denote by $\flux$ the flux of $\Phi$ across $\bregion\omega$. Then, the evolution of $\Phi$ can be modeled by an \lingo{integral balance law}:
  %
  \begin{equation}\label{eqn:integral-balance-law}
    \odt t\int_{\omega}\phi\,\dvol
    + \oint_{\bregion\omega}\flux\iprod\dsurf
    =
    \pm \int_{\omega}\ssink\,\dvol\,,
  \end{equation}
  %
  where $+\ssink,-\ssink$ represent a \lingo{source density of $\Phi$} and a \lingo{sink density of $\Phi$}.
\end{observation}
%
The integral balance law can be stated in words:~\cite[p. 5]{mishra2016}
%
\begin{law}
  for a conserved quantity moving through a body, the quantity temporal change plus its flux across the body surface equals the total amount of the quantity being consumed or generated inside the body.
\end{law}

\begin{model}
  Consider a physical quantity of density $\phi$ move through a body, as in \cref{obs:balance-law-conserved-quantity}. Then, the evolution of $\phi$ can be modeled by a \lingo{differential balance law in Euler's form}:
  %
  \begin{equation}\label{eqn:differential-balance-law-euler}
    \dt\phi + \div\flux = \pm\ssink\,.
  \end{equation}
\end{model}
%
\begin{argument}
  Depart from \cref{eqn:integral-balance-law}. Differentiate under the volume integral, apply then the divergence theorem to the area integral, gather the resulting integrands into a volume integral, and finally apply the localization to it.
\end{argument}
%
By changing the notation of the derivatives, \cref{eqn:differential-balance-law-euler} can be recasted into traditional notation:
%
\begin{equation}\label{eqn:differential-balance-law-euler-operator}
  \cpd t\phi + \odiv\flux = \pm\ssink\,.
\end{equation}

The advantage of abstract balance laws, like \cref{eqn:integral-balance-law,eqn:differential-balance-law-euler}, over other approaches lies in their generality:~\cite{thorne2013,mishra2016}
%
\begin{law}
  once a conserved quantity has been chosen, we can derive its \scare{continuity equation} by inserting the relevant physics into the quantity flux $\flux$ and then evaluate its resulting contribution to the balance law. At each step, one gets out, in the form $\div\flux$, the physics that one puts into $\flux$.
\end{law}
%
Notice the efficiency with which we can generate specialized versions of balance laws.
%
\footnote{For instance, see how laboriously the \scare{continuity equation of mass}, a specialized version of \cref{eqn:differential-balance-law-euler}, is deduced in~\cite[p. 42]{holzbecher2012}.}

\begin{theorem}
  Consider a conserved quantity of density $\phi$ be transported through a body flowing with velocity $\vel$. Then, the evolution of $\phi$ can be modeled by a \lingo{differential balance law in Lagrange's form}:
  %
  \begin{equation}\label{eqn:differential-balance-law-lagrange}
    \mdt\phi + \phi\div\vel + \div\naflux = \pm\ssink\,,
  \end{equation}
  %
  where $\naflux$ represents the \lingo{non-advective} part of the flux.
\end{theorem}
%
\begin{proof}
  Depart from a differential balance law, as \cref{eqn:differential-balance-law-euler-operator}:
  %
  \begin{equation*}
    \cpd t\phi + \odiv\flux = \pm\ssink\,.
  \end{equation*}
  %
  Since $\flux$ is represented by a vector, separate it into an advective part $\phi\vel$ and a non-advective part $\naflux$, such that $\flux = \phi\vel + \naflux$. Replace this into the last equation to have
  %
  \begin{equation*}
    \cpd t\phi + \odiv\br{\phi\vel + \naflux} = \pm\ssink\,,
  \end{equation*}
  %
  which by the linearity of the divergence becomes
  %
  \begin{equation*}
    \cpd t\phi + \odiv\br{\phi\vel} + \odiv\naflux = \pm\ssink\,.
  \end{equation*}
  %
  Expand the divergence by means of its product rule:
  %
  \begin{equation*}
    \cpd t\phi + \vel\iprod\ograd\phi + \phi\odiv\vel + \odiv\naflux = \pm\ssink\,.
  \end{equation*}
  %
  Express the first two terms as the material derivative of $\phi$:
  %
  \begin{equation*}
    \mdt\phi + \phi\odiv\vel + \odiv\naflux = \pm\ssink\,.
  \end{equation*}
\end{proof}


\section{Mass balance laws}
\label{sec:mass-balance-laws}

\begin{model}
  Consider a body deform. Suppose the body be represented by a fluid flowing with velocity $\vel$. Then, the evolution of the fluid mass density can be modeled by a \lingo{mass conservation law in Lagrange's form}:
  %
  \begin{equation}\label{eqn:mass-conservation-law-lagrange}
    \mdt\dens + \dens\div\vel = 0\,.
  \end{equation}
  %
\end{model}
%
\begin{argument}
  Since the fluid mass is conserved, the mass is transported only due to advection, and there are neither mass sources nor sinks, plug in the fluid density $\dens$ and its (advective) flux, $\dens\vel$, into a differential balance law, like \cref{eqn:differential-balance-law-lagrange}, with vanishing $\ssink$ term.
\end{argument}

\begin{definition}
  Consider a fluid flow. Call the flow \lingo{incompressible} if and only if the fluid mass density changes neither on space nor time; \ie, $\mdt\dens = 0$.
\end{definition}

\begin{model}
  Consider a fluid deforms with under incompressible flow. Then, the evolution of the fluid velocity can be modeled by an \lingo{incompressible flow equation}:
  %
  \begin{equation}\label{eqn:incompressible-flow-equation}
    \div\vel = 0\,.
  \end{equation}
  %
\end{model}
%
\begin{argument}
  Since the fluid mass is conserved, the mass is transported only due to advection, and there are neither mass sources nor sinks, plug in the fluid density $\dens$ and its (advective) flux, $\dens\vel$, into a differential balance law, like \cref{eqn:mass-conservation-law-lagrange}, with vanishing $\ssink$ term and apply the definition of incompressible flow.
\end{argument}
%
The last model states in words that
%
\begin{law}
  whenever a fluid flows under incompressible flow, the divergence of its velocity vanishes.
\end{law}


\section{Momentum balance laws}
\label{sec:momentum-balance-laws}

\begin{model}
  Consider a fluid flow under incompressible flow and under a gravitational field of free-fall acceleration $\grav$. Then, the fluid acceleration can be modeled by a \lingo{Cauchy's momentum equation in Lagrange's form}:
  %
  \begin{equation}\label{eqn:cauchys-momentum-equation-lagrange}
    \mdt\vel + \spvol\div\stress = \grav\,,
  \end{equation}
  %
  where $\spvol,\stress$ represent the fluid \lingo{specific volume} and \lingo{stress}.
\end{model}
%
\begin{argument}
  Since the fluid momentum is conserved, depart from a differential balance law, as \cref{eqn:differential-balance-law-lagrange}. Recall that the fluid momentum density is $\dens\vel$ and that the stress $\stress$ is defined as the momentum flux. Note additionally that gravitational field acts as an external source of momentum, $\dens\grav$:
  %
  \begin{equation*}
    \mdt{\dens\vel} + \dens\vel\tprod\div\vel + \div\stress = \dens\grav\,.
  \end{equation*}
  %
  Expand the material derivative:
  %
  \begin{equation*}
    \dens\mdt\vel + \vel\mdt\dens + \dens\vel\tprod\div\vel + \div\stress = \dens\grav\,.
  \end{equation*}
  %
  Now, since the flow is incompressible, $\mdt\dens$ and $\div\vel$ vanish (the latter statement is a courtesy of the incompressible flow equation, \cref{eqn:incompressible-flow-equation}):
  %
  \begin{equation*}
    \dens\mdt\vel + \div\stress = \dens\grav\,.
  \end{equation*}
  %
  Multiply finally by the fluid specific volume.
\end{argument}

Cauchy's momentum equation, \cref{eqn:cauchys-momentum-equation-lagrange}, comes in handy frequently when developing relations in fluid mechanics:~\cite[chap. 13, p. 19]{thorne2013}
%
\begin{law}
  as we add new pieces of physics to our fluid analysis (isotropic pressure, viscosity, gravity, magnetic forces), an efficient way to proceed at each stage is to insert the relevant physics into the stress tensor $\stress$, and then evaluate the resulting contribution $\div\stress$ to Cauchy's momentum equation, \cref{eqn:cauchys-momentum-equation-lagrange}. That is, \emph{at each step, we get out in the form $\div\stress$ the physics that we put into $\stress$}.
\end{law}
